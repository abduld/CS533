\subsection{Question 1:}\label{question-1}

a.

\begin{enumerate}
\def\labelenumi{\arabic{enumi}.}
\item
  Redundant multi-threaded is able to detect transient errors even in
  the context of SMT. In CMP, then the RMT is able to detect permanent
  errors, but you cannot guarantee that you can correct the error. SMT
  is not able to detect permanent errors, since two threads can reuse
  the same structural unit.
\item
  ReEnact can detect data races and possibly correct them (by providing
  hints to the programmer). Synchronization operations create epochs
  (which use Lamport vector clocks) which provide ordering use to detect
  data races. ReEnact allows one to cleanly undo groups of tasks,
  re-execute those tasks, and replay the tasks deterministically even
  when run in parallel.
\item
  ReVive detects transient and permanent errors in one node's memory or
  transient errors in many nodes. The entire memory is protected by a
  distributed parity (similar to RAID5) and periodically sets
  checkpoints, which writes the cache line back to memory and saves the
  processor context. This only requires changes to the directory
  controller.
\end{enumerate}

\begin{enumerate}
\def\labelenumi{\alph{enumi}.}
\setcounter{enumi}{1}
\itemsep1pt\parskip0pt\parsep0pt
\item
  Even if you know the system state, you do not know the scheduling
  information. Because determining data races requires one to to
  consider all possible locations where shared variables are accessed
  across threads, the problem becomes untraceable.
\end{enumerate}

c.

Partial ordering is enforced using vector clocks, an epoch terminates
when it reaches a synchronization point and a new epoch is generated.
The newly generated epoch's ID is a successor of the terminated one.
Upon aquire synchronization, the next epoch ID is set to be the
successor of the ID of the most recenet release synchronization. Upon
commit or communication, IDs of pairs of epochs are compared to make
sure they are in correct order (other wise a race occurred). If an epoch
is executed and one less than it is executed at a later stage, then the
sequential semantics of the program is violated and one has to roll back
and re-execute the code.

\begin{enumerate}
\def\labelenumi{\alph{enumi}.}
\setcounter{enumi}{3}
\itemsep1pt\parskip0pt\parsep0pt
\item
  Small epochs cause large overhead, since one has to copy registers,
  but is more robust to detecting error. Having large epochs means that
  many statements are executed between epochs and therefore you cannot
  roll back
\end{enumerate}

Many epochs cause many copies results in the overhead of epoch creation
being noticeable and can cause the same cache line and these can map to
the same bank and therefore use up its associativity. This results in
both a large memory overhead and a higher cache miss rate.

The $cautious$ configuration uses small epochs to allow one to roll back
the history a bit more. This degrades performance. The $balanced$
configuration tries to find a $balanced$ configuration where one
balances the roll back window with the performance impact.

\subsection{Question 2:}\label{question-2}

a.

They both suffer from instruction cache misses, and therefore a larger
instruct cache size increases performance. Since both perform
computation on memory, some queries can map onto memory on chip
architectures.

Increasing clock frequency would not help in OLTP and DSS because the
performance is dominated by memory operations, but the CPI is low enough
that it does not make a difference.

\begin{enumerate}
\def\labelenumi{\arabic{enumi}.}
\item
  OLTP has simple read/write queries that must be serviced online. Both
  suffer from many dirty misses and therefore a write through coherence
  protocol may result in better performance (since dirty cache access
  can be serviced quicker).
\item
  DSS has complicated queries that are read only. Changing your
  coherence protocol, to exploit the fact that memory can never be
  overwritten, can decrease network traffic.
\end{enumerate}

b.

\begin{enumerate}
\def\labelenumi{\arabic{enumi}.}
\item
  Information gathering overhead which is caused by cache miss
  collection. To reduce overhead, the paper uses sampling, counting only
  one in 10 cache misses. For larger systems, grouping is used to group
  processors into logical units and sharing a counter across them. The
  second source of overhead is kernel and data movement overhead. They
  streamline the communication between CPUs and collect multiple pages
  before TLB flushes. TLB flushes can be reduced by tracking which pages
  are mapped between processors. The third source of overhead is
  replication space overhead. They only replicate hot pages and
  replicate code on first touch.
\item
  The performance would be much worse, because the misses are too coarse
  grained. Therefore if you sample the TLB misses, then you may not be
  able to capture the cache miss behavior.
\item
  The answer is in 2b1
\end{enumerate}

\subsection{Question 3:}\label{question-3}

\begin{enumerate}
\def\labelenumi{\alph{enumi}.}
\itemsep1pt\parskip0pt\parsep0pt
\item
  A shared memory multiprocessor is a shared memory multiprocessor
  connected via an interconnect where one can add extra resources (and
  achieve expected performance gains) regardless of the size of the
  system. This means that there is no dependence between the performance
  gain of adding a resource and the size of the system. Typically this
  is unrealistic, since as the system gets large then other resources
  come into effect.
\end{enumerate}

b.

\begin{enumerate}
\def\labelenumi{\arabic{enumi}.}
\item
  A torus that's embedded onto a path network. The network is
  unidirectional and one dimensional, and allows each node to
  communicate with another.
\item
  A 1D torus with bidirectional links, a linear area, a 2D torus with
  each path being unidirectional, or a collection of bipartite graphs
  each with bidirectional links (this is called a d-dimensional 2-arry
  butterfly in the book). And any combination of them is feasible. In
  fact, any directed connected graph with out degree $2$ and in degree
  $2$.
\end{enumerate}

\begin{enumerate}
\def\labelenumi{\alph{enumi}.}
\setcounter{enumi}{2}
\item
  Since you have a non-minimal routing algorithm, you have many paths
  between nodes. Collecting state information requires one to add more
  network bandwidth and more complexity to the switches. The switches
  now cannot use simple heuristics or rules to compute the minimal
  route, and therefore need to be either table based or source based. If
  the topology changes or the network changes, then the switches need to
  be adaptive. In general, even with state information, taking a minimal
  path is a known problem does not scale, since time complexity of
  minimal path is $O(VE)$ (computed via Bellamn-Ford) with $V$ and $E$
  are the numbers of the verticies and edges.
\item
  The two are equivalent when there is no traffic or when one paclet is
  equivalent to one flit.
\item
  Since message passing requires on to send messages to other nodes, the
  header contains source and destination information which are not
  necessary on a single node. One can make use of the fact that
  processors use write-invalidate cache by implementing the MPI protocol
  to (rather than send a copy of the data to the other process) one can
  share the data with the other process in read only mode. If this is
  not exploited, then multiple copies of the data need to be passed
  between processes.
\end{enumerate}

f.

\begin{enumerate}
\def\labelenumi{\arabic{enumi}.}
\item
  Deadlock is accomplished by using logically independent requests and
  response networks are supported with two virtual channels each.
\item
  The routing algorithm is minimal because it use dimension order and it
  uses a 3D torus topology.
\item
  The routing algorithm is not adaptive because it use a fixed routing
  path --- dimension order.
\end{enumerate}

\subsection{Question 4:}\label{question-4}

\begin{enumerate}
\def\labelenumi{\alph{enumi}.}
\item
  The compiler performs analysis to determine blocks of code which
  operate on data and encodes instructions as a dataflow graph. The EDGE
  ISA would not contain the register information, rather the edges of
  the source data. TRIPS, for example, does not contain the source
  operands. The instruction would only be executed if its dependent
  instructions are executed.
\item
  Programmer can only think about a few variables and the same time, and
  therefore reuse those variables (creating dependencies). Dependencies
  between regions rarely exist (or are very little) and therefore are
  amicable to be run in parallel. Modern super scalers have a large
  instruction window (100 instructions) to allow the CPU to determine
  parallelism via register renaming and dependence determination.
\end{enumerate}

If you have a dataflow machine then you do not rely on the instruction
window, an can make use of all available parallelism in your program
(this is equivalent to having an infinite instruction window). It is up
to the compiler and architecture to able to perform whole program
analysis to determine dependence information and execute instructions in
parallel. Fine grained synchronization is explicit in the dataflow
graph.

\begin{enumerate}
\def\labelenumi{\alph{enumi}.}
\setcounter{enumi}{2}
\itemsep1pt\parskip0pt\parsep0pt
\item
  Since infinite instructions windows are not possible to implement, one
  has to limit the instruction window. TRIPS implements a dual-core 16
  wide issue processor with a 1024 instruction window. And, unlike VLIW,
  where one has different types of units (integer or floating point)
  TRIPS has no distinction. Each block in TRIPS is executed in a
  dataflow maner (encoded in VLIW style in the implementation), but
  connection between the blocks is von Neumann style. TRIPS generates
  code specific to a particular hardware. Ash, unlike TRIPS, is not
  implementation dependent, generating IR code (Pegasus) that is then
  compiled to specific architecture. WaveScalar generates both RISC-like
  ISA along with dependence information, these are scheduled in waves
  dynamically. All are von Neuman machines that must somehow enable one
  to simulate the big instruction window.
\end{enumerate}
