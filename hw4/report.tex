\subsection{Question 1:}

a.

\begin{enumerate}
\def\labelenumi{\arabic{enumi}.}
\item
  Redundant multi-threaded is able to detect transient errors even in
  the context of SMT. In CMP, then the RMT is able to detect perminant
  errors, but you cannot guarantee that you can correct the error. SMT
  is not able to detect perimentant errors, since two threads can reuse
  the same structural unit.
\item
  ReEnact can detect data races and possibly correct them (by providing
  hints to the programmer). Sychronization operations create epochs
  (which use Lamport vector clocks) \ldots{}
\end{enumerate}

TODO

\begin{enumerate}
\def\labelenumi{\arabic{enumi}.}
\setcounter{enumi}{2}
\itemsep1pt\parskip0pt\parsep0pt
\item
  ReVive detects transient and perminant errors in one node's memory or
  transient errors in many nodes.
\end{enumerate}

TODO

\begin{enumerate}
\def\labelenumi{\alph{enumi}.}
\setcounter{enumi}{1}
\itemsep1pt\parskip0pt\parsep0pt
\item
  Even if you know the system state, you do not know the scheduling
  information. Because determining data races requires one to to
  consider all possible locations where shared variables are accessed
  across threads, the problem becomes untractable.
\end{enumerate}

c.

TODO

\begin{enumerate}
\def\labelenumi{\alph{enumi}.}
\setcounter{enumi}{3}
\itemsep1pt\parskip0pt\parsep0pt
\item
  Small epochs cause large overhead, since one has to copy registers,
  but is more robust to detecting error. Having large epochs means that
  many statements are executed between epochs and therefore you cannot
  roll back
\end{enumerate}

Many epochs cause many copies results in the overhead of epoch creation
being noticiable and can cause the same cache line and these can map to
the same bank and therefore use up its associativity. This results in
both a large memory overhead and a higher cache miss rate.

The $cautious$ configuration uses small epochs to allow one to roll back
the history a bit more. This degrades performance. The $balanced$
configuration tries to find a $balanced$ configuration where one
balances the roll back window with the performance impact.

\subsection{Question 2:}

a.

They both suffer from instruction cache misses, and therefore a larger
instruct cache size increases performance. Since both perform
computation on memory, some queries can map onto memory on chip
architectures.

Increasing clock frequency would not help in OLTP and DSS because the
performance is dominated by memory operations, but the CPI is low enough
that it does not make a difference.

\begin{enumerate}
\def\labelenumi{\arabic{enumi}.}
\item
  OLTP has simple read/write queries that must be serviced online. Both
  suffer from many dirty misses and therefore a write through coherence
  protocol may result in better performance (since dirty cache access
  can be serviced quicker).
\item
  DSS has complicated queries that are read only. Changing your
  coherence protocol, to exploit the fact that memory can never be
  overwritten, can decrease network traffic.
\end{enumerate}

b.

\begin{enumerate}
\def\labelenumi{\arabic{enumi}.}
\item
  Information gathering overhead which is caused by cache miss
  collection. To reduce overhead, the paper uses sampling, counting only
  one in 10 cache misses. For larger systems, grouping is used to group
  processers into logical units and sharing a counter across them. The
  second source of overhead is kernel and data movement overhead. They
  streamline the communication between CPUs and collect multiple pages
  before TLB flushes. TLB flushes can be reduced by tracking which pages
  are mapped between processors. The third source of overhead is
  replication space overhead. They only replicate hot pages and
  replicate code on first touch.
\item
  The performance would be much worse, because the misses are too coarse
  grained. Therefore if you sample the TLB misses, then you may not be
  able to capture the cache miss behavior.
\item
  The answer is in 2b1
\end{enumerate}

\subsection{Question 3:}

\begin{enumerate}
\def\labelenumi{\alph{enumi}.}
\itemsep1pt\parskip0pt\parsep0pt
\item
  A shared memory multiprocessor is a shared memory multiprocessor
  connected via an interconnect where one can add extra resources (and
  achieve expected performance gains) regardless of the size of the
  system. This means that there is no dependence between the performance
  gain of adding a resource and the size of the sytem. Typically this is
  unrealistic, since as the system gets large then other resources come
  into effect.
\end{enumerate}

b.

\begin{enumerate}
\def\labelenumi{\arabic{enumi}.}
\item
  A torus that's embeded onto a path network. The network is
  unidirectional and one dimensional, and allows each node to
  communicate with another.
\item
  A 1D torus with bidirectional links, a linear area, a 2D torus with
  each path being unidirectional, or a collection of bipratite graphs
  each with bidirectional links (this is called a d-dimensional 2-arry
  butterfly in the book). And any combination of them is feasable. In
  fact, any directed connected graph with out degree $2$ and in degree
  $2$.
\end{enumerate}

\begin{enumerate}
\def\labelenumi{\alph{enumi}.}
\setcounter{enumi}{2}
\item
  Since you have a non-minimal routing algorithm, you have many paths
  between nodes. Collecting state information requires one to add more
  network bandwidth and more complexity to the switches. The switches
  now cannot use simple heuristics or rules to compute the minimal
  route, and therefore need to be either table based or source based. If
  the topology changes or the network changes, then the switches need to
  be adpative. In general, even with state inforamtion, taking a minimal
  path is a known problem does not scale, since time complexity of
  minimal path is $O(VE)$ (computed via Bellamn-Ford) with $V$ and $E$
  are the numbers of the verticies and edges.
\item
  The two are equivalent when there is no traffic or when one paclet is
  equivalent to one flit.
\item
  Since message passing requires on to send messages to other nodes, the
  header contains source and destination information which are not
  nessary on a single node. One can make use of the fact that processors
  use write-invalidate cache by implementing the MPI protocol to (rather
  than send a copy of the data to the other process) one can share the
  data with the other process in read only mode. If this is not
  exploited, then multiple copies of the data need to be passed between
  processes.
\end{enumerate}

f.

\begin{enumerate}
\def\labelenumi{\arabic{enumi}.}
\item
  Deadlock is acomplished by using logically independent requests and
  response networks are supported with two virtual channels each.
\item
  The routing algorithm is minimal because it use dimension order and it
  uses a 3D torus toplogy.
\item
  The routing algorithm is not adaptive because it use a fixed routing
  path --- dimension order.
\end{enumerate}

\subsection{Question 4:}

\begin{enumerate}
\def\labelenumi{\alph{enumi}.}
\item
  The compiler perfroms analysis to determine blocks of code which
  operate on data and encodes instructions as a dataflow graph. The EDGE
  ISA would not contain the register information, rather the edges of
  the source data. The instruction would only be executed if its
  arguments are executed.
\item
  If you have a dataflow machine then you do not rely on the instruction
  window, an can make use of all available parallism in your program
  (this is equivalent to having an infinite instruction window). It is
  upto the compiler and architecture to able to perform whole program
  analysis to determine dependence information and execute instructions
  in parallel. Fine grained synchronization is explicit in the dataflow
  graph.
\item
  For trips, atomic instruction blocks are dataflow machines, but the
  connection between these blocks are similar to von Neumann.
\end{enumerate}
