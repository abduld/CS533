%-----------------------------------------------------------------------------
%
%               Template for sigplanconf LaTeX Class
%
% Name:         sigplanconf-template.tex
%
% Purpose:      A template for sigplanconf.cls, which is a LaTeX 2e class
%               file for SIGPLAN conference proceedings.
%
% Guide:        Refer to "Author's Guide to the ACM SIGPLAN Class,"
%               sigplanconf-guide.pdf
%
% Author:       Paul C. Anagnostopoulos
%               Windfall Software
%               978 371-2316
%               paul@windfall.com
%
% Created:      15 February 2005
%
%-----------------------------------------------------------------------------


\documentclass[nocopyrightspace]{sigplanconf}

% The following \documentclass options may be useful:
%
% 10pt          To set in 10-point type instead of 9-point.
% 11pt          To set in 11-point type instead of 9-point.
% authoryear    To obtain author/year citation style instead of numeric.

\usepackage{natbib}
\usepackage{url}
\usepackage[squaren,thinqspace,binary]{SIunits}

\let\fourth\relax               % Undefine a macro
\let\second\relax               % Undefine a macro
\let\degree\relax               % Undefine a macro
\let\cdot\relax                 % Undefine a macro
\usepackage{array}
\usepackage{amsmath}
\usepackage{amssymb}
\usepackage{graphicx}
%\usepackage{mathabx}
\usepackage{mathtools}
\usepackage{multirow}
\usepackage{bussproofs}
\usepackage{verbatim}
\usepackage{fancyvrb}

\hyphenation{mono-morph-izing}
\hyphenation{mono-morph-ize}
\hyphenation{mono-morph-ic}
\hyphenation{poly-morph-ic}
\hyphenation{geo-mean}

%------------------------------------------------------------------------------
% Global macro definitions

\newcommand{\abs}[1]{\lvert#1\rvert}
\newcommand{\proportional}{\mathbin{\propto}}
\newcommand{\inst}{\mathrel{\succeq}}
\newcommand{\defeq}{\mathrel{:=}}
\newcommand{\defeqcont}{\mathrel{\hphantom{\defeq}\mathllap{\mid}}}
\newcommand{\overlineX}[1]{\overline{\vphantom{\beta}#1}}
\newcommand{\overlinet}[1]{\overline{\vphantom{t}#1}}
\newcommand{\conexprX}[3]{#1\app\overlineX{#2}\app\overlineX{#3}}
\newcommand{\conexpr}[3]{#1\app\overline{#2}\app\overline{#3}}
\newcommand{\dataexpr}[4]{#1\app\overline{#2}\app\overline{#3}\app\overline{#4}}
\newcommand{\subst}[2]{[#2/#1]}
\newcommand{\substo}[2]{[\overline{#2/#1}]}

\newcommand{\ztype}[1]{\ensuremath{#1_{\text{type}}}}
\newcommand{\name}[1]{\ensuremath{\mathsf{#1}}} % Object functions
\newcommand{\meta}[1]{\ensuremath{\mathrm{#1}}} % Meta-functions
\newcommand{\bmk}[1]{\texttt{#1}}
\newcommand{\app}{\;}

\def\T{\mathbin{::}}

\newcommand{\qforall}[1]{\forall#1.\nolinebreak[1]\:}
\newcommand{\qexists}[1]{\exists#1.\nolinebreak[1]\:}
\newcommand{\qlambda}[1]{\lambda#1 \to}
\newcommand{\qLambda}[1]{\Lambda#1 \to}
\newcommand{\qforallT}[2]{\qforall{#1\T#2}}
\newcommand{\qexistsT}[2]{\qexists{#1\T#2}}
\newcommand{\qlambdaT}[2]{\qlambda{#1\T#2}}
\newcommand{\qLambdaT}[2]{\qLambda{#1\T#2}}

\newcommand{\letE}[1]{\text{\bf let}\;#1\;\text{\bf in}\;}
\newcommand{\letrecE}[1]{\text{\bf letrec}\;#1\;\text{\bf in}\;}
\newcommand{\letfE}[1]{\text{\bf letfun}\;#1\;\text{\bf in}\;}
\newcommand{\caseE}[1]{\text{\bf case}\;#1\nolinebreak[1]\;\text{\bf of}\;}
\newcommand{\dataE}[1]{\text{\bf data}\nolinebreak[1]\;#1\nolinebreak[0]\;\text{\bf where}\;}
\newcommand{\classE}[1]{\text{\bf class}\nolinebreak[1]\;#1\nolinebreak[0]\;\text{\bf where}\;}
\newcommand{\ifE}[3]{\ifO#1\;\thenO#2\;\elseO#3}
\newcommand{\externO}{\text{\bf extern}\;}
\newcommand{\caseO}{\text{\bf case}\;}
\newcommand{\dataO}{\text{\bf data}\;}
\newcommand{\classO}{\text{\bf class}\;}
\newcommand{\ifO}{\text{\bf if}\;}
\newcommand{\thenO}{\text{\bf then}\;}
\newcommand{\elseO}{\text{\bf else}\;}
\newcommand{\whereO}{\text{\bf where}\;}
\newcommand{\ofO}{\text{\bf of}\;}
\newcommand{\letO}{\text{\bf let}\;}
\newcommand{\letrecO}{\text{\bf letrec}\;}
\newcommand{\inO}{\text{\bf in}\;}
\newcommand{\typeO}{\text{\bf type}\;}

\def\kindstar{\ensuremath{\mathord{\star}}}
\def\kindZ{\ensuremath{\mathbb{Z}}}
\def\kindbox{\name{box}}
\def\kindbare{\name{bare}}
\def\kindval{\name{val}}
\def\kindout{\name{mut}}
\def\kindinit{\name{init}}


\def\typeOut{\name{Mut}}
\def\typeStore{\name{Store}}
\def\typeInit{\name{Init}}
\def\typeStored{\name{Stored}}
\def\typeBoxed{\name{Boxed}}
\def\typeAsBox{\name{AsBox}}
\def\typeAsBare{\name{AsBare}}
\def\typeArr{\name{Arr}}
\def\typeRef{\name{Ref}}
\def\typeRep{\name{Rep}}
\def\typeTuple{\name{Tuple}}
\def\typeTupleV{\name{TupleV}}
\def\typeList{\name{List}}
\def\typeInt{\name{Int}}
\def\typeFloat{\name{Float}}
\def\typeBool{\mathbf{2}}
\def\typeMaybe{\name{Maybe}}
\def\typeScatter{\name{Sc}}
\def\typeSz{\name{Sz}}
\def\typeSv{\name{Sv}}
\def\typeZ{\name{Z}}
\def\typePtr{\name{Ptr}}

\def\contuple{\name{tuple}}
\def\constored{\name{stored}}
\def\conboxed{\name{boxed}}
\def\conref{\name{ref}}
\def\conrepr{\name{rep}}
\def\consize{\name{size}}
\def\conasBare{\name{asBare}}
\def\conasBox{\name{asBox}}
\def\concopy{\name{copy}}
\def\conreprInt{\name{repInt}}
\def\conreprTuple{\name{repTuple}}
\def\coniint{\name{z}}
\def\coneiint{\name{exz}}
\def\consz{\name{sz}}
\def\consv{\name{sv}}
\def\conz{\name{z}}
\def\contrue{\name{true}}
\def\confalse{\name{false}}

\newcommand{\Unit}{\langle\mspace{0.75mu}\rangle}
\newcommand{\Prod}[1]{\langle#1\rangle}
\newcommand{\Array}[1]{\mathopen{\text{\sf[}}#1\mathclose{\text{\sf]}}}

\def\iterIdx{\name{Idx}}
\def\iterStep{\name{Step}}
\def\iterFold{\name{Fold}}
\def\iterColl{\name{Coll}}
\def\iterIter{\name{Iter}}

\newcommand{\todo}[1]{\textbf{[#1]}}

\def\proofsep{13pt}

\newcommand{\proofspacing}{%
  \parindent=0cm%
  \everypar={\hskip 0cm plus 1fill}%
  \parfillskip=0cm plus 1fill\relax%
  \parskip=\proofsep}

\newenvironment{inlinecodeexample}%
{\begin{tabbing}}
{\end{tabbing}}

\newcommand{\setwherestretch}{\def\arraystretch{1.15}}
\newcommand{\where}{\hphantom{=\mathord{}}\mathllap{\text{where}}}
\newenvironment{whereblock}[1]%
{\setwherestretch\where\begin{array}[t]{#1}}%
{\end{array}}

%------------------------------------------------------------------------------

\begin{document}

%\titlebanner{banner above paper title}        % These are ignored unless
%\preprintfooter{preprint}   % 'preprint' option specified.

\title{ZOne}
%\subtitle{}

\authorinfo{Abdul Dakkak \and Li-Wen Chang \and Carl Pearson}
           {University of Illinois at Urbana-Champaign}
           {\{dakkak, lchang20, pearson\}@illinois.edu}

\conferenceinfo{CONF 'yy}{Month d--d, 20yy, City, ST, Country} 
\copyrightyear{20yy} 
\copyrightdata{978-1-nnnn-nnnn-n/yy/mm} 
\doi{nnnnnnn.nnnnnnn}

\maketitle

%\category{CR-number}{subcategory}{third-level}

%\terms
%term1, term2

%\keywords
%keyword1, keyword2

\section*{Motivation and Overview}
Hello from intro.tex
\cite{placeholder}


\section{Related Work}

This work that is related to our project emphasizes expression of 
data-parallelism through higher-order functions and high-level languages. In
this section, we present a selection of related works.

DryadLINQ\cite{yu2008dryadlinq} is a high-level language for data-parallel
computing. It is designed to handle batch operations on large-scale
distributed systems. It combines strongly-typed .NET objects,
general-purpose declarative and imperative statements, and LINQ expressions
into a sequential program that can be debugged with a standard .NET debugger.
The DryadLINQ system automatically transforms the data-parallel portions of the
program into a distributed execution plan.
% TODO: more to write?

Triolet\cite{rodrigues2014triolet} is a programming interface and system for
distributed-memory clusters that handles task decomposition, scheduling, and
communication. Triolet emphasizes algorithmic skeletons to capture parallel
computation patterns and abstract away the details of the implementation.
Triolet presents the programmer with higher-order functions through which
expressed parallelism is automatically distributed across a cluster.

Thrust\cite{thrust} is a parallel algorithms library for C++ resembling the C++
standard library. Thrust code can use CUDA, Intel TBB, or OpenMP as a final
target to enable high performance across a variety of systems.

Copperhead\cite{copperhead} is a data-parallel version of a subset of Python
that is
interoperable with traditional Python code and Numpy. Copperhead can use CUDA,
Intel TBB, and OpenMP to accelerate operations. The Copperhead runtime
intercepts annotated function calls and transforms them (and their input data)
to the appropriate
CUDA, OpenMP, or TBB constructs before executing them.

Accelerate\cite{accelerate} is an embedded array language in Haskell for 
high-performance
computing. It allows computations on multi-dimensional arrays to be expressed
through collective higher-order operations such as maps or reductions. It has
a CUDA and OpenCL backend.

NOVA\cite{collins2013nova} is a data-parallel polymorphic, statically-typed
functional language. Parallelism is expressed through higher-order functions
such as scan, reduce, map, permute, gather, slice, and filter. NOVA has a
squential/parallel C backend, and a CUDA backend.

Adaptive Implementation Selection in SkePU\cite{enmyren2010skepu} is a C++ template library for data-parallel
computations on one or more GPUs through CUDA or OpenCL. SkePU programs are
expressed through skeletons derived from higher-order functions. Notably,
SkePU also implements lazy memory copying to avoid unecessary memory transfers.

GPU MapReduce (GPMR)\cite{stuart2011multi} is a  MapReduce library written
for multi-GPU clusters. GPMR programs are expressed through the map and reduce
higher-order functions. GPMR breaks these programs up into a map stage, a sort stage, and a reduce stage, then uses optimizations that reduce communication at
the expense of computation, which is a tradeoff that is well-suited for GPUs.
These optimizations are all based off of the data partitioning that the
programmer selects.


% We recommend abbrvnat bibliography style.

\bibliographystyle{abbrvnat}
\bibliography{paper}

% The bibliography should be embedded for final submission.

%\begin{thebibliography}{}
%\softraggedright
%
%\bibitem[Smith et~al.(2009)Smith, Jones]{smith02}
%P. Q. Smith, and X. Y. Jones. ...reference text...
%
%\end{thebibliography}

\end{document}
