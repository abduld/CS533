
\subsubsection{Intel Threaded Building Blocks}
Intel Threaded Building Blocks\cite{reinders2007intel} (TBB) is a library for
scalable parallel
programming in C++. It provides templates for common parallel programming
patterns and abstracts away the details of synchronization, load-balancing,
and cache optimization. Instead of writing threads the programmer specifies
tasks, and the library maps them onto threads in an efficient manner.

We use TBB to perform parallel file I/O as well as invoking the
	\fix{cudaMalloc} call (since it does not have an asynchronus version)
	in a background thread.
Since some parts of the code need to be performed atomically, 
	setting error or modifying the list of memory objects used, for example,
	each function takes a \fix{zState\_t} object as its first argument.
The state object contains all ``global'' variables and locks needed to 
	safely modify state visable from other threads.


When the state object is created, we create a set of mutexes that
	are to be reused (a logger, error, timer, etc\ldots mutexes).
We then use a macro to allow us to easily write these mutexes:

\begin{verbatim}
#define zState_mutexed(lbl, ...)            \
  do {                                      \
    speculative_spin_mutex mutex =          \
    zState_getMutex(st, zStateLabel_##lbl); \
    mutex::scoped_lock();                   \
    { __VA_ARGS__; }                        \
  } while (0)
\end{verbatim}


Throughout our code, we use the above macro to update our state.
To set an error, for example, we just write \fix{zState\_mutexed(Error, zState\_setError(st, memoryAllocation))}.


\fix{speculative\_spin\_mutex} are used to decrease overhead due to locks.
\fix{speculative\_spin\_mutex}es are a lock concept in TBB
built on top of transactional memory. The locks and unlocks in
\fix{speculative\_spin\_mutex} are marked as eligible for elision, meaning that
the lock is not modified in memory and the thread continues until it has
reached the unlock. Any changes that the thread makes to memory will not be
initially visible to other processors. If no conflicts occur during that time,
the changes are atomically committed to memory.

A conflict could either take the form of a memory location being updated, or
the lock being taken non-speculatively by another processor. To avoid false
sharing, a \fix{speculative\_spin\_mutex} takes two cache lines. This
guarantees that at least one cache line boundary will exist within the object,
so the actual lock portion of the class can fill up an entire cache line by
itself. Presently, \fix{speculative\_spin\_mutex} will only execute
speculatively on 4th-generation ``Haswell'' Intel Core parts with TSX enabled.
