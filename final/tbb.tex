
\subsubsection{Intel Threaded Building Blocks}
Intel Threaded Building Blocks\cite{reinders2007intel} (TBB) is a library for
scalable parallel
programming in C++. It provides templates for common parallel programming
patterns and abstracts away the details of synchronization, load-balancing,
and cache optimization. Instead of writing threads the programmer specifies
tasks, and the library maps them onto threads in an efficient manner.

We use TBB to perform parallel file I/O as well as invoking the
	\fix{cudaMalloc} call (since it does not have an asynchronus version)
	in a background thread.
Since some parts of the code need to be performed atomically, such as
	setting error codes or modifying the list of memory objects used,
	each function takes a \fix{zState\_t} object as its first argument.
The state object contains all ``global'' variables and locks needed to
	safely modify state visible from other threads.


When the state object is created, we create a set of mutexes that
	are to be reused (a logger, error, timer, etc\ldots mutexes).
We then use a macro to allow us to easily write these mutexes:

\begin{verbatim}
#define zState_mutexed(st, lbl, ...)        \
  do {                                      \
    speculative_spin_mutex::scoped_lock(    \
      zState_getMutex(st, zStateLabel_##lbl)\
    );                                      \
    __VA_ARGS__;                            \
  } while (0)
\end{verbatim}


Throughout our runtime code, we use the above macro to update our state ---
 	for example, setting the error of the program to \fix{memoryAllocate} is done via \fix{zState\_mutexed(st, Error, zState\_setError(st, memoryAllocate))}.


Based on ...., we use \fix{speculative\_spin\_mutex} to increase performance.
In ....
\todo[inline]{Why we use \fix{speculative\_spin\_mutex}}


