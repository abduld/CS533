\section*{Evaluation}

\subsection*{Benchmarks}
To demonstrate the performance and correctness of our ZOne implementation, we
presently use three benchmarks: 2D convolution, histogram, and conjugate
gradient.

\subsubsection*{Convolution}
Convolutions have many uses in engineering and mathematics, particularly in
the image-processing fields. High-performance CPU convolution imlpementations
involve vectorization and tiling to make full use of cache bandwidth and 
execution resources. The ZOne convolution code is shown below

\begin{verbatim}
such code    wow

   impress
                speed
\end{verbatim}


\subsubsection*{Histogram}
Histograms are a fundamental analysis tool in image and data processing.
Efficient serial CPU histogram implementations are very straghtforward, but
due to the data-dependant access pattern efficient GPU implementations are
more involved. They typically feature privitization,
where individual histograms for portions of the data are computed separately
and then compiled together into the overall result. This reduces serialization
of atomic memory accesses when different threads increment the same bin.
Other approaches include sorting the input data and then finding the start
index of each bucket, and approaches that use graphics-specific hardware like
occlusion queries.

\begin{verbatim}
such code    wow

   impress
                speed
\end{verbatim}


\subsubsection*{Conjugate Gradient}
The congugate gradient method is an algorithm for solving systems of linear
equations. It is usually implemented as an iterative algorithm for sparse
systems that are too large to use a direct implementation.

\begin{verbatim}
such code    wow

   impress
                speed
\end{verbatim}

