%\subsubsection{The Zone Language}
%\label{sec:zonelang}

Our statically typed language is inspired by array and data flow
programming languages such as Fortran~\cite{Fortran}, APL~\cite{APL}, and
LINQ~\cite{LINQ} where one expresses computation based on operations on
vectors and arrays. In our language, one can define two vectors of size
\fix{100} by

\begin{verbatim}
n :: Integer = 1000;
as :: []Real = rand.Real(n);
bs :: []Real = rand.Real(n); 
\end{verbatim}


The syntactic style for ZOne is borrowed from a other languages
	which we think have good syntax representations.
The type declaration position was browed from Go~\cite{golang},
	the function declarations from Dart~\cite{dartlang},
	the type notation from Julia~\cite{julialang},
	and the semantics from array languages such as Fortran~\cite{Fortran} or APL~\cite{APL}.
The result is a unique language that is applicable to wide array
	of applications.



To give you a taste of the language, in this program we approximate
\fix{pi} using Monte Carlo integration

\begin{verbatim}
def f(a :: Real, b :: Real) :: Bool {
    return a*a + b*b < 1;
}
res :: Integer = zip(as, bs).count(f) / n;
\end{verbatim}

this would be translated into the map/reduce operations of

\begin{verbatim}
t1 = map(f, zip(as, bs));
count = 0;
reduce((x) => count += x, t1);
res = count / n;
\end{verbatim}

The compiler would lower the above into an IR representation,
maintaining the map/reduce operations. This is discussed 
in more detail later in the report.


The language is purely functional, disallowing any side effects.
It is also statically typed with type anotation provided by the user.
Due to time constraints, it lacks many features such as conditionals.
This limitation can be avoided through simple bit operations.


