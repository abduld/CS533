\subsubsection{The Zone Language}


The ZOne language borrows different programming textual styles that
	the we found useful in other languages.
The type declaration position was browed from Go \todo[inline]{cite},
	the function declarations from Dart \todo[inline]{cite},
	the type notation from Julia \todo[inline{cite}],
	and the semantics from array languages such as Fortran, APL,
	and ..\todo[inline]{cite}.
The result is a unique language that is applicable to wide array
	of applications.
In one of our benchmarks, for example, we use it to define the
	Black-Scholes model which can be expressed in Dart by:


\begin{verbatim}
TODO FILL ME
\end{verbatim} 


The language is purely functional --- disallowing any side effects.
It is also statically typed --- with type anotation provided by the user.
Due to time constraints, it lacks many features such as conditionals, but 
	one can go around that limitation using simple bit operations.
