%\subsubsection{Parser}

Using parser combinators, we are able to simplify the definition of 
	the language (compared to traditional \fix{LR(1)} parsers).
The parser is expressed within dart through a Dart library and
	the parser and lexer are in the same file (as compared to two
	files as traditionally used by \fix{lex/yacc}~\cite{lexyacc}).
The library also provides convenient primitives, such as a definition of
	a floating point numbers, to simplify our definition.
Furthermore, it provides utility functions to declare common patterns
	in a convenient manner.
To express function parameters, for example, first define how one function
	parameter is to be parsed:

\begin{verbatim}
param() =>
    identifier + typeIdentifier ^
        (id, type) => new ParameterNode(id, type); 
\end{verbatim}

Then, we use the built-in utility function \fix{sepBy}, \fix{parens},
	and \fix{comma} to define how a collection of parameters are
	to be parsed

\begin{verbatim}
params() => parens(param().sepBy(comma));
\end{verbatim}

While using parser combinator is cleaner, it does take longer to correct
	that traditional $LR$ parsers --- left recursions, for example, are
	not allowable in parser combinators and one must restructure the parser
	to go around this limitation.
