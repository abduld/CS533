\subsubsection{Parser}

Using parser combinators, we are able to simplify the definition of 
	the language (compared to traditional \fix{LR(1)} parsers).
Using a Dart library, we are able to express the parser within dart and
	have the parser and lexer in the same file (as compared to two
	files as traditionally used by \fix{lex/yacc}).
The library also provides convinient primitives, such as a definition of
	a floating point numbers, to simplify our definition.
Futhermore, it provides us with utility function to declare common patterns
	in a convinient manner.
To express function paramaters, for example, first define how one function
	parameter is to be parsed:

\begin{verbatim}
param() =>
	identifier + typeIdentifier ^
		(id, type) => new ParameterNode(id, type); 
\end{verbatim}

Then, we use the builtin utility function \fix{sepBy}, \fix{parens},
	and \fix{comma} to define how a collection of parameters are
	to be parsed

\begin{verbatim}
params() => parens(param().sepBy(comma));
\end{verbatim}

While using parser combinator is cleaner, it does take longer to get correctly
	that traditional $LR$ parsers --- left recursions, for example, are
	not allowable in parser combinators and one must restructure the parser
	to go around this limitation.
