\section{Future Work}

While ZOne demonstrates some important features of an effective parallel
programming environment, there is plenty of opportunity to improve the
applicability and efficiency of the code generation.

Support for OpenCL, a heterogeneous computing framework for writing programs
that run
on CPUs, GPUs, DSPs, FPGAs, and other available computing resources, would be a
logical step for the runtime. OpenCL is inspired by CUDA and has stream-like
capabilities, so many of the CUDA constructs that ZOne can
generate code for can easily be retargeted to OpenCL to allow ZOne to generate
efficient code for devices that are not NVIDIA GPUs. 

It is worth mentioning that our aim was not to write an optimized
compiler (the compiler can be very slow), our aim was to have a compiler
that can generate optimized code. To that end, we have written the
compiler in Dart (a Javascript inspired language) that currently
generates sequential Javascript and threaded CUDA code from our language.
The Javascript generation is primarily meant for debugging purposes.
A robust compiler would require more optimizations passes, such as data layout
transformation and vectorization. The compiler built for this project is only
complicated enough to show that high-level languages can be mapped onto the
runtime - a useful compiler would have to be extended with more advanced
optimization passes.
