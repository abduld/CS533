
\subsubsection{CUDA Streams}
By design, CUDA device kernels execute asycnhronously with respect to the
host code. This allows host code to overlap with device code, but does not
allow different device operations (kernels and data transfers) to execute
concurrently. CUDA exposes device concurrency through \textit{CUDA Streams}
\cite{kirk2012programming}.

A CUDA Stream is a sequence of operations that
execute in-order on a CUDA device. Separate streams may be interleaved or
overlap if possible. In this way, it is possible to overlap computation and
memory transfers to hide the latency of some operations.
In order to effectively use streams, CUDA allows synchronization operations
to occur between arbitrary streams and provides host code that allows the
CPU to determine the execution status and progress of different streams.

The flexibility of CUDA streams is limited by the device hardware.
For example, the Fermi architecture can manage one queue of kernels, one queue
of device$\rightarrow$host transfers, and one queue of host$\rightarrow$device
transfers. Stream
dependencies between queues are maintained, but there are no dependencies
within queues - the operations are simply done in the order they are put into
the queue. This allows overlap of data transfer and compute on Fermi GPUs.
On devices that support more than one concurrent compute operation the amount
of concurrency is limited by the execution resources on the device. If both
operations are too large to run concurrently they will be serialized.


CUDA streams are used for our copy operations and we register callbacks via the \fix{cudaStreamAddCallback} function to set flags and clear mutexes once
the asynchronus requrest completes.
These flags and mutexes are checked before a kernel launch or copy to either	make sure the data is available before a computation, or that data has
	already been copied and does not need to be recopied.

	