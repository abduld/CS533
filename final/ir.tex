\subsubsection{Intermediate Representation}

For the MonteCarlo implementation shown previously, the 
	compiler generates the following IR.

\begin{verbatim}
Instruction(zab, zip(as, bs))
Instruction(t1, Map(f, zab))
Instruction(count, 0)
Function(g, (x) => count += x)
Instruction(r2, Reduce(g, t1))
Instruction(res, divide(count, n))
\end{verbatim}

There are a few things to notice about the IR.
First, while it is similar to $3$-address form,
	we represent it as an $S$-expression internally 
	(with a head defining what the instruction is and a body
	as a list of arguments).
Second, \fix{Map} and \fix{Reduce} operations are preserved in the IR, 
	allowing us to keep high level information.
Finally, binary operations, such as $+$ and $-$, are not
	treated specially and are simply function calls.

The compiler is able to analyze the above code, inline the map operation
into the reduce function, and privatize the \fix{count} variable. This
results in efficient code that can be parallelized and does not produce
unnecessary temporary arrays.
