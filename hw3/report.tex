\subsection{Question 1: Prefetching}\label{question-1-prefetching}

\begin{verbatim}
    (1) for i = 0 to 127 
    (2)     for j = 0 to 31 
    (3)         A[i,j] = B[i] * C[j] 
\end{verbatim}

\begin{enumerate}
\def\labelenumi{\alph{enumi}.}
\itemsep1pt\parskip0pt\parsep0pt
\item
  You are going to get a cache miss $3*(128*32*4)/16 = 3*1024$ times, so
  the number of cycles wasted due to cache misses is $3*1024 * 64$
  cycles = $3*2^16$ cycles. So the above loop would take
\end{enumerate}

\begin{gather*}
128*32*16 \text{(cyles when their is a cache hit)} +\\
      3*2^16 \text{(cycles for memory fetch due to cache miss)} +\\
      129*33*n \text{where $n$ is however many cycles to execute the branch and conditionals}
\end{gather*}

\begin{enumerate}
\def\labelenumi{\alph{enumi}.}
\setcounter{enumi}{1}
\item
  It is clear the we need to prefetch data $4$ elements away for each
  aray (e.g. \texttt{C{[}j+4{]}}) since the cache line is $16$ bytes and
  each element in the array is $4$. We can prefetch the \texttt{B} array
  in the outer loop (since it is only dependent on the \texttt{i}
  induction variable), since we will have a cold miss in the first loop
  iteration, we can service that prefetch before \texttt{2a} and
  \texttt{2b} for the next \texttt{i} iteration. After the cold miss, we
  prefetch \texttt{C} and \texttt{A} for the next \texttt{j} iteration.

\begin{verbatim}
(1) for i = 0 to 127 
(1p)    PREFETCH(&B[i+4])
(2)     for j = 0 to 31
(3)         A[i,j] = B[i] * C[j] 
(2a)        PREFETCH(&C[j+4])
(2a)        PREFETCH(&A[i,j+4])
\end{verbatim}
\end{enumerate}

\subsection{Question 2 : Prefetching 2}\label{question-2-prefetching-2}

\begin{enumerate}
\def\labelenumi{\alph{enumi}.}
\itemsep1pt\parskip0pt\parsep0pt
\item
  Prefetching tackles the problem of cold cache misses
\item
  In a single issue processor you can only have one instruction in
  flight, so a prefetch latency would not be hidden in place of a cold
  miss. In a multiple issue processor, you can hide the prefetch
  latency.
\item
  When you are in the critical section, otherwise you cannot assume that
  the memory location (which you've now bound a register to) has not
  been modified by another processor.
\item
  No, since once you have the prefetched data in cache then the cache
  coherency model would work the same. Care must be taken however to
  make sure that you invalidate before you write, otherwise you can
  prefetch a memory location while another processor is invalidating and
  writing to it. In binding prefetches, one adds special logic for this
  issue.
\item
  In a long latency cache miss enter runahead mode. In runahead mode,
  speculatively execute instructions to generate prefetches. When
  original miss returns, you exit runahead mode. The intuition is that
  this generates some accurate prefetch information and would hide some
  of the latency of the original cache miss.
\item
  One can use the cores as memory prefetch units. So you offload some of
  the loads and stores onto them and use them as a closer memory fetch
  unit (and their L1 cache as a slightly farther away L1 cache).
\end{enumerate}

\subsection{Question 3 : Synchronization
1}\label{question-3-synchronization-1}

\begin{enumerate}
\def\labelenumi{\alph{enumi}.}
\item
  This would change the semantic of the algorithm since the check
  \texttt{bar\_name.counter == p} is currently within the atomic block.
  To see the problem consider the interleaving

\begin{verbatim}
Thread 1                           Thread 2
UNLOCK(bar_name.lock);
                                   LOCK(bar_name.lock);
                                   mycount = bar_name.counter++;
if (bar_name.counter == p)
\end{verbatim}
\end{enumerate}

\texttt{Thread 1}'s condition would fail and would go into an infinite
loop.

\begin{enumerate}
\def\labelenumi{\alph{enumi}.}
\setcounter{enumi}{1}
\item
  Yes

\begin{verbatim}
BARRIER(bar_name, p) {
    local_sense = !local_sense
    if fetch_and_inc(&bar_name.counter) == p {
        bar_name.counter = 0
        bar_name.flag = local_sense
    } else {
        while (bar_name.flag != local_sense) {}
    }
}
\end{verbatim}
\item
  You still need locks, since different threads might enter the same
  critical sections if the OS performs context switching inside the
  section. Consider the following timeline

\begin{verbatim}
                            Thread 1        Thread 2
> start critical section      i1              
                              i2 (long latency)
> os decides to context switch              i1
                                            i2
\end{verbatim}
\end{enumerate}

this is clearly wrong. And to implement this you would still need an
atomic operation (again to guarantee that the instruction sequence get
executed in order).

\begin{enumerate}
\def\labelenumi{\alph{enumi}.}
\setcounter{enumi}{3}
\item
  \texttt{compare\_and\_swap} is implemented as:

\begin{verbatim}
compare_and_swap(p, old, new) {
    BEGIN_ATOMIC()
    if(*p == old)
        *p = new
    END_ATOMIC()
    return old;
}
\end{verbatim}
\end{enumerate}

Using \texttt{LL} and \texttt{SC} you can implement
\texttt{compare\_and\_swap} as

\begin{verbatim}
    compare_and_swap(p, old, new) {
        do {
            x = LL(p)
        } while(!SC(x == old ? old : new, p))
        return old;
    }
\end{verbatim}

\texttt{fetch\_and\_inc} is implemented as

\begin{verbatim}
    fetch_and_inc(loc) {
        BEGIN_ATOMIC()
        x = *old
        *loc = x + 1
        END_ATOMIC()
        return x;
    }
\end{verbatim}

Using \texttt{LL} and \texttt{SC} you can implement
\texttt{fetch\_and\_inc} as

\begin{verbatim}
    fetch_and_inc(loc) {
        do {
            x = LL(loc)
            y = x + 1
        } while(!SC(loc, y))
        return x;
    }
\end{verbatim}

\subsection{Question 4 : Synchronization
2}\label{question-4-synchronization-2}

\begin{enumerate}
\def\labelenumi{\alph{enumi}.}
\item
  Here \texttt{CH} is a cache hit, \texttt{TS} is the test and set
  operation, and \texttt{WR} is write back.

\begin{verbatim}
p1     p2  ... p8
INV    INV     INV
RM     RM      RM
TS     CH      CH
WR     RM      RM
       RM      RM
       TS      CH
       WR      RM
           ... TS
\end{verbatim}
\end{enumerate}

From the above, we can figure out that there is
$3 + 4 + 5 + ... + (8 + 3) = 60$ bus transactions.

\begin{enumerate}
\def\labelenumi{\alph{enumi}.}
\setcounter{enumi}{1}
\item
  There is no guarantee that the first processor would get serviced
  first, so since there are $60$ bus transactions then $300$ cycles
  elapse before all $8$ processors get serviced. So, on average $P1$
  would take $\frac{300}{8} = 37.5 \text{ cycles}$.
\item
  If there is no cache then test-test-and-set would still generate less
  traffic than test-and-set, since test-and-set is essentially an atomic
  read/write while in test-test-and-set we can perform the read without
  the need for the write in many iterations. So, you'd save a bit less
  traffic by not sending data into memory, and just reading from it.
\item
  Yes, recall that one can implement either using \texttt{LL} and
  \texttt{SC} (previous question).
\end{enumerate}

\subsection{Question 5 : CMT and SMT}\label{question-5-cmt-and-smt}

\begin{enumerate}
\def\labelenumi{\alph{enumi}.}
\item
  SMT differs from superscalers by providing parallelism at the thread
  level rather than at the instruction level, and it different from
  traditional multithreading by using the same core to execute the
  instructions from different threads simultaneously.
\item
  The structures
\end{enumerate}

\begin{itemize}
\itemsep1pt\parskip0pt\parsep0pt
\item
  Branch Predictor --- Augmented with a thread id for SMT. By augmenting
  with the thread id, you'd increase your prediction accuracy
\item
  Return Address Stack --- Duplicate
\item
  Register File --- Shared
\item
  TLB --- Shared
\item
  RAT --- Shared
\item
  LSQ --- Shared
\end{itemize}

\begin{enumerate}
\def\labelenumi{\alph{enumi}.}
\setcounter{enumi}{2}
\item
  As you increase your issue width you will increase the rename, wakeup,
  and bypass delay. This is due to the increase in wire length.
\item
  For CMP these are traditional independent cores with a coherence
  model. Some hardware is also needed for CMP to allow it to communicate
  registers with other processors. SMT requires one to duplicate certain
  hardware structures, such as the PC, return address stack, etc\ldots{}
  this increases the cost and latency of certain operations. The
  advantage of SMT is that you can avoid some structural hazards that
  you'd otherwise get in a uniprocessor machine.
\item
  A task may get squashed either because its predecessor gets squashed
  or if detects that a load and a store conflicts (did not happen in
  program order). Subsequent tasks also get squashed.
\end{enumerate}
