\subsection{Question 1: TLS}

\begin{enumerate}
\def\labelenumi{\alph{enumi}.}
\item
  TLS attempts at providing a form of auto-parallization (with some
  support from the compiler). The compiler insearts check points, and
  code is executed speculatively (possibly the next iteration of the
  loop). If reach the end of a checkpoint and the speculative execution
  is not squashed (via memory writes, exceptions, \ldots{}) then its
  state is committed. Otherwise, you squash the thread and roll the
  state back to the point of the checkpoint and resume speculative
  execution. TLS requires hardware support.
\item
  For one, ILP does not work across cores, but TLS can. One of the main
  disadvantages of ILP is that (usually) a sequence of program
  statements have true (RAW) dependencies on one another, but a sequence
  of loop iterations do not have this dependencies (in a doAll loop for
  example). TLS is also able to parallize larger chunks of loop
  iterations with the help of the compiler.
\end{enumerate}

c.

\begin{enumerate}
\def\labelenumi{\arabic{enumi}.}
\item
  If you squash, then the instruction count increases. Otherwise, your
  instruction count stays the same (ignoring the instructions to commit
  and save the state).
\item
  The memory system's performance does not change, you maybe transfering
  more data on the bus and may evict some cache lines due to capcity or
  conflict, but memory performance is not impacted very much.
\item
  In some cases the predictors need to be duplicated. It might make
  sense, for example, not to polute the branch predictor if you are
  speculatively executing and only commit the modifications to the
  predictor if the thread is not squashed. With duplication, then you
  may encure some extra cycles to load and save state.
\end{enumerate}

d.

\begin{enumerate}
\def\labelenumi{\alph{enumi}.}
\setcounter{enumi}{4}
\itemsep1pt\parskip0pt\parsep0pt
\item
  I would look at the implementation and figure out the dependencies. In
  some cases, one can just place OpenMP pragmas and have a
  multi-threaded implementation. In other cases (for example in large
  code bases), the dependencies are not obvious and one cannot
  auto-parallize the code. In those cases one has to determine the
  programming effort and make a judgment on whether the returns justify
  the effort. TLS may aid the compiler/programming in generating
  auto-parallizable code by speculatively assuming that there are no
  dependencies between loop iterations.
\end{enumerate}

f.

g.

\subsection{Question 2 : Speculative Synchronization}

\begin{enumerate}
\def\labelenumi{\alph{enumi}.}
\item
  Transactional memory allows regions memory operations to be executed
  atomically. This is equivalent to grabing a lock, executing the
  instructions, and then releasing the lock in software. In TLS, this
  would mean saving the state, executing the instructions, and if not
  squashed committing the results.
\item
  Strong maintains atomicity between transactional and non-trasactional
  regions while weak only maintains atomicity between transactional
  regions.
\end{enumerate}

c.

\begin{verbatim}
1. TLS attempts to provide a way to execute

2. TM 
\end{verbatim}

d.

\begin{verbatim}
1.

2. The output commit problem states that you cannot undo buffers sent to the I/O device. 
Let's say a thread outputs to a screen, if you execute the thread speculatly, then this means that you cannot undo the output.
A way around this is to buffer the I/O operations and only commit them at
a checkpoint.

3. 
\end{verbatim}

e.

\subsection{Question 3 : Processor in Memory}

\begin{enumerate}
\def\labelenumi{\alph{enumi}.}
\item
  PIM brings computation close to the memory system and therefor can
  give you high compute throughput. It allows you to not fetch data from
  memory iff the PIM has the capability to execute the instruction. The
  main disadvantage is in terms of manifacturing. Memory is manifactured
  to increase space density, while processors are manifactured to
  increase compute speed. There is also the basic question of how useful
  are PIMs in the first place, usual programs have the processor usually
  fetching memory and execute hundreds or thousands of instructions on
  the memory fetched, this means that the memory copy overhead is not as
  high as the compute overhead.
\item
  The type of applications that benifit from PIM are ones that do basic
  operations or permutations on the memory elements (transpose, shuffle,
  etc\ldots{}). High compute work loads, such as scientific workloads,
  would not work on PIM (unless you add support for those instructions
  --- which results in you just having a second processor that's closer
  to memory).
\item
  The PIM is faster in all cases.

\begin{verbatim}
1) for (i = 0; i < 1000; i++) 
     A[i] = B[i]; 
\end{verbatim}
\end{enumerate}

There is a cache miss every 8 iterations. On the CPU, there will be
$100*ceil(1000/8) + 1000*1 = 13500cycles$ this means the operations
complete in $13.5us$. On the PIM, there will be
$4*ceil(1000/8) + 1000 = 1500cycles$ this means the operations complete
in $7.5us$.

\begin{verbatim}
    2) for (i = 0; i < 1000; i++) 
         A[i] = B[i] * B[i + 1]; 
\end{verbatim}

There is a cache miss every 7 iterations. On the CPU, there will be
$100*ceil(1000/7) + 2*1000 = 16300cycles$ this means the operations
complete in $16.3us$. On the PIM, there will be
$4*ceil(1000/7) + 2000 = 1286cycles$ this means the operations complete
in $12.86us$.

\begin{verbatim}
    3) for (i = 0; i < 1000; i += 4) 
         A[i] = B[i]; 
\end{verbatim}

There is a cache miss every 2 iterations. On the CPU, there will be
$100*ceil(1000/(4*2)) + 1000/4 = 1275cycles$ this means the operations
complete in $12.75us$. On the PIM, there will be
$4*ceil(1000/(4*2)) + 1000/4 = 3750cycles$ this means the operations
complete in $3.75us$.

\subsection{Question 3 : Reliability}
